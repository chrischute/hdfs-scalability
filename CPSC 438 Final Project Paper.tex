% CPSC 438 Final Project Paper
% Christopher Chute, David Brandfonbrener, Leo Shimonaka, Matthew Vasseur

% Author and Title Information
\newcommand*{\thetitle}{Scalability Limits of HDFS}
\newcommand*{\theauthor}{Christopher Chute, David Brandfonbrener, Leo Shimonaka, Matthew Vasseur}
\newcommand*{\duedate}{May 2, 2016}

% Document Settings
\documentclass[11pt, a4paper]{article}
\author{\theauthor}
\title{\thetitle}
\date{\duedate}
\usepackage[top=.75in, left=0.5in, right=0.5in, bottom=1.5in]{geometry}
\usepackage{amsmath, amsthm, amssymb}
\newtheorem{lem}{Lemma}
\usepackage{graphicx}
\usepackage{setspace}
\usepackage{enumitem}
\usepackage{booktabs}
\usepackage[T1]{fontenc}
\usepackage{titling}

% Multicolumn Formatting
\usepackage{multicol}
\setlength\columnsep{18pt}    % spacing between columns

\setlength{\droptitle}{-9em}
\posttitle{\par\end{center}\vspace{-.35em}}

% Header Formatting
\usepackage{fancyhdr}
\setlength{\headheight}{48pt}
\pagestyle{fancyplain}
%\lhead{\thetitle}
%\rhead{\theauthor\\\duedate}
%\rfoot{}
\cfoot{\thepage}

% ************************* End of Preamble ***********************
\begin{document}
\maketitle
\thispagestyle{empty}

\begin{abstract}
% TODO: Complete after paper done.
\end{abstract}
\begin{multicols*}{2}

% -------------------------------------------------

\section{Introduction}

% Introduce HDFS and the notion of Distributed FileSystems, their prevalence, etc
% I.e. Some B.S. on what's out there to set the background

% -------------------------------------------------

\section{HDFS Architecture}

% Describe the 5 Hadoop design principles (Perhaps move to Introduction)

% Describe typical HDFS cluster: Single Namenode, Multiple Datanodes
% The goal for decoupling metadata from actual data is scalability

% Benefits of Single Namenode Architecture:
% 1) Given an operation, all namespace operations run on single Namenode while expensive data transfers are distributed across Datanodes of cluster
% 2) Simple durability scheme via replication of the same data block across Datanodes. 
% 3) Same replication technique over independent Datanodes also enable a highly available system exploiting principles of MapReduce.

\subsection{Namenode}

% The entire metadata namespace is kept in memory. 
% In-memory namespace keeps operations from external clients (get_block_locations, create_block) and internal communication (heartbeats, block reports) fast
% File that maps UNIX-style inodes to data block locations in Datanodes
% Write-Ahead log in stable storage

\subsection{Datanode}

% Stores file's data blocks
% Periodic heartbeat to notify Namenode of its health (i.e. if it's running)
% Periodic block reports to notify Namenode of a stored block's status (e.g. block write recieved)



% Perhaps add section for how Read / Write operations are processed in HDFS? Might help describe the type of operations Namenode has to process.
% E.g. Client Request Workflow for reading data

% -------------------------------------------------

\section{Limitations of Single Namenode Architecture}

\subsection{Physical Memory Size}

% Re-iterate in-memory metadata namespace
% Total File Capacity of Cluster is Limited by Name Node's RAM due to requirement that all metadata objects (file inodes and block) in memory at all times

% SVENCHO PAPER EXAMPLE: Assuming on average, a file consists of \lambda = 1.5 blocks, then each file uses 1 file and ~2 data block objects, hence uses ~600 bytes. Thus to keep 100 million files, name-node must have at least 60GM of RAM


\subsection{Namenode CPU Bottleneck}

% Describe the INTERNAL-LOAD communicating with Datanodes
% # of Heartbeats that Namenode recieves is directly proportional to # of Datanodes in cluster
% # of Block reports is also directly proportional to size of data blocks (i.e. ratio of blocks mapped per file) and also # of replicated copies per block
% Physical data storage and data I/O performance increase proportionally to # of Datanodes in cluster and size of data blocks, but the overall performance of HDFS is also affected negatively by increased internal load.
% The more time spend on internal load, less time spend on processing external client's requests


% Describe the external-load from clients
% All metadata operations run on Namenode - they cannot be processed anywhere else.
% On every file read, there is a get_block_location per block per file
% On every file write, there is a create_block per block per file
% Namenode can become a bottleneck with operations that involve many metadata operations, e.g. batch-write operations will include many create_block requests

% -------------------------------------------------

\section{Performance Evaluation}

% Discuss goals of tests (i.e. "we want to test Memory and CPU and determine blah")

% Mention experimental setup (EC2, Java Test Skeleton, # nodes, etc)

% FOR EACH EXPERIMENT:
% Discuss designs of individual experiments and their results
% What were the experiments testing?
% What were the different parameters of the experiments?
% Possibilities of error?

\subsection{Experiment 1: Memory Experiments}

% TODO: See above

\subsection{Experiment 2: CPU Experiments}

% TODO: See above

\subsection{Analysis}

% Discuss raw results independently for each experiment:
% Are these results what was expected (and why)?
% If there was some error, discuss what may have caused it
% Graphs

% Evaluate experimental results holistically to pinpoint the problem with HDFS's Single Namenode architecture (I.e. weigh CPU and Memory problems against each other)

% -------------------------------------------------

\section{Improvements to Single Namenode Architecture}

% Use experimental data to determine what is actually detrimental to performance of HDFS, and suggest a way to improve it

% -------------------------------------------------

\section{Conclusion}

% Wrap it up with stuff.





\end{multicols*}
\end{document}


























